
\section{Результаты}

Приложения Controller и приложение для тестирования Fake Service Discovery были написаны на языке Java 17 с использованием фреймворка Spring и  библиотек io.envoyproxy.controlplane, io.grpc. Скрипт для создания шаблонной конфигурации Envoy был написан на языке bash. И приложения echo-server и service для тестирования были написаны на языке Python с использованием flask.

\section{Заключение}

В рамках данной работы проводилось исследование существующих подходов к реализации Service Mesh и на основе рассмотренных подходов была предложена архитектура Service Mesh для 1С облака. Также была реализовано приложение Controller, входящие в Service Mesh Control Plane и набор инструментов, позволяющих с легко настраивать Service Mesh Data Plane.

При дальнейших исследованиях планируется проверить колличество потребляемых ресурсов данного приложения при большей нагрузке и при негативных результатах можно рассматривать такие улучшения как смена транспортного протокола Envoy на то, чтобы в рамках запроса на обновление конфигурации не происходила полная переконфигурация, а только разница между текущей конфигурацией и новой. Также очередь запросов может переполниться и может потребоваться вынести ее в отдельное приложение, такое как Kafka или RabbitMQ.


Другим вектором для будущих исследований может быть добавление новой функциональности для Service Mesh, например, обеспечение безопсности.
%Имплементация описанного выше решения была написана на C++20 с использованием библиотек Eigen, Function2, GLFW3, Cxxopts и Vulkan. Исходный код имеет размер порядка 10000 строк и доступен по ссылке \url{https://github.com/Mrkol/thesis_vgi}.
%
%В качестве тестовых моделей были выбраны 3D-сканы Quixel Megascans \cite{quixel_megascans} в максимальном качестве. В следующей таблице приведено время работы этапов предобработки, а также количество полигонов и название использованных моделей. В столбец "подготовка" объединены этапы (1-2), в столбец "кластеризация" этапы (3-4), в "квадрангуляцию" -- (5-6), а последние 2 столбца соответствуют этапам (7) и (8). В разных строках указаны запуски с разной целевой ошибкой кластеризации, в столбце "кластеры" указано итоговое количество кластеров. Замеры происходили на Intel Core i5-470 (3.20Ghz) и видеокарте NVIDIA GeForce RTX 2070 SUPER.
%
%
%\begin{table}[ht]
%\small
%\centering
%\begin{tabular}{ l | p{16mm} | p{15mm} | p{17mm} | p{12mm} | p{14mm} | p{13mm} | p{17mm} }
%                        & Полигоны (млн.) & Кластеры & Подготовка (мин.) & Кластер. (мин.) & Квадранг. (мин.) & Репарам. (мин.) & Ресемплинг (мин.) \\
%\hline
%\multirow{2}{18mm}{Rock Cliffs (uchwaffda)}
%& 3    & 130 & 2.1   & 0.8  & 40   & 105    & 0.2   \\
%& 3    & 164 & 2.1   & 0.8  & 40   & 83     & 0.2   \\
%\hline
%\multirow{3}{16mm}{Rock Sandstone (rlbx3)}
%& 5    & 30  & 3.4   & 1.4  & 116  & 700    & 0.5   \\
%& 5    & 224 & 3.5   & 1.4  & 34   & 200    & 0.3   \\
%&&&&&&&\\
%\end{tabular}
%\caption{Производительность алгоритма предобработки}
%\end{table}
%
%Как видно из таблицы, время затраченное на репараметризацию сильно зависит от количества кластеров. Более того, при малом числе кластеров их форма как правило становится достаточно иррегулярной, что приводит к значительно большему времени на этапе квадрангуляции из-за необходимости делить рёбра. С другой стороны большее число кластеров приводит к более медленному рендерингу из-за необходимости адаптировать б\'ольшее число квадродеревьев.
%
%\begin{figure}[h]
%  \minipage{0.45\textwidth}
%    \begin{tikzpicture}
%      \begin{axis}[
%          scale=0.85,
%          ybar,
%          symbolic x coords={Минимум,Среднее,Максимум},
%          xtick=data,
%          enlarge x limits=0.2,
%          legend style={at={(0.025,0.975)}, anchor=north west},
%          ylabel={FPS}
%          ]
%        \addplot[ybar,fill=yellow] coordinates {
%          (Минимум,250) (Среднее,290) (Максимум,420)
%        };
%        \addplot[ybar,fill=red] coordinates {
%          (Минимум,200) (Среднее,260) (Максимум,682)
%        };
%        \addplot[ybar,fill=blue] coordinates {
%          (Минимум,24) (Среднее,26) (Максимум,457)
%        };
%        \legend{Уровни детализации,Предлагаемый метод,Исходная модель}
%      \end{axis}
%    \end{tikzpicture}
%    \caption{Производительность рендеринга модели ``Rock Cliffs (uchwaffda)'' в кадрах в секунду (FPS, frames per second). Больше -- лучше}
%    \label{fig:uchwaffda_perf}
%  \endminipage\hfill
%%
%  \minipage{0.45\textwidth}
%    \begin{tikzpicture}
%      \begin{axis}[
%          scale=0.85,
%          ybar,
%          symbolic x coords={Минимум,Среднее,Максимум},
%          xtick=data,
%          enlarge x limits=0.2,
%          legend style={at={(0.025,0.975)}, anchor=north west},
%          ylabel={FPS}
%          ]
%        \addplot[ybar,fill=yellow] coordinates {
%          (Минимум,130) (Среднее,140) (Максимум,250)
%        };
%        \addplot[ybar,fill=red] coordinates {
%          (Минимум,103) (Среднее,140) (Максимум,309)
%        };
%        \addplot[ybar,fill=blue] coordinates {
%          (Минимум,21) (Среднее,23) (Максимум,403)
%        };
%        \legend{Уровни детализации,Предлагаемый метод,Исходная модель}
%      \end{axis}
%    \end{tikzpicture}
%    \caption{Производительность рендеринга модели ``Rock Assembly Rough (sjzbj)'' в кадрах в секунду (FPS, frames per second). Больше -- лучше}
%    \label{fig:sjzbj_perf}
%  \endminipage\hfill
%\end{figure}
%
%Производительность и качество рендерига же измерялись на процессоре Intel Core i7-8550U и его встроенной видеокарте. Производительность алгоритма при движении камеры показана на рисунках \ref{fig:uchwaffda_perf}-\ref{fig:sjzbj_perf}. Для сравнения были взяты исходная высокополигональная модель отрендеренная и предоставляемые Quixel её уровни детализации. Из графиков видно, что использование иерархического атласа не сильно ухудшает производительность по сравнению с наивным использованием уровней детализации, но при этом в разы быстрее рендеринга полной исходной модели.
%
%Противопоставим эти показатели итоговому качеству картинки. Для его измерения была использована метрика SSIM \cite{wang2004image}, разработанная специально для сравнения видимого качества семантически одинаковых изображений. Сравниваемые изображения, а также изображения каркасов соответствующих моделей, представлены на рисунках \ref{fig:preview1start}-\ref{fig:preview3end}. Изображения полученные при помощи разработанного решения, а также полученные с помощью наивных уровней детализации, были сопоставлены с исходной высокополигональной моделью с трёх разных ракурсов. Результаты представлены на рисунке \ref{fig:quality}. Таким образом разработанное решение позволяет добиться качества на 20 процентных пунктов лучше наивного подхода, при этом незначительно проигрывая ему в производительности.
%
%\newcommand{\showpreviews}[1]{
%  \begin{figure}[h]
%    \minipage{0.32\textwidth}
%      \centering
%      \includegraphics[scale=0.079]{high#1}
%      \caption{Оригинальная модель. Вид с ракурса #1}
%      \label{fig:preview#1start}
%    \endminipage\hfill
%    \minipage{0.32\textwidth}
%      \centering
%      \includegraphics[scale=0.079]{ours#1}
%      \caption{Предлагаемый метод. Вид с ракурса #1}
%    \endminipage\hfill
%    \minipage{0.32\textwidth}
%      \centering
%      \includegraphics[scale=0.079]{lod#1}
%      \caption{Уровни детализации. Вид с ракурса #1}
%    \endminipage
%    \\
%    \minipage{0.32\textwidth}
%      \centering
%      \includegraphics[scale=0.079]{high#1_wireframe}
%      \caption{Оригинальная модель. Вид с ракурса #1 (сетка)}
%    \endminipage\hfill
%    \minipage{0.32\textwidth}
%      \centering
%      \includegraphics[scale=0.079]{ours#1_wireframe}
%      \caption{Предлагаемый метод. Вид с ракурса #1 (сетка)}
%    \endminipage\hfill
%    \minipage{0.32\textwidth}
%      \centering
%      \includegraphics[scale=0.079]{lod#1_wireframe}
%      \caption{Уровни детализации. Вид с ракурса #1 (сетка)}
%      \label{fig:preview#1end}
%    \endminipage
%  \end{figure}
%}
%
%\showpreviews{1}
%\showpreviews{2}
%\showpreviews{3}
%
%\begin{figure}[h]
%  \centering
%  \begin{tikzpicture}
%    \begin{axis}[
%        width=0.8\textwidth,
%        height=0.6\textwidth,
%        ybar,
%        ymin=0,ymax=1,
%        symbolic x coords={Разработанное решение,Уровни детализации},
%        xtick=data,
%        enlarge x limits=0.2,
%        ylabel={SSIM},
%        ]
%      \addplot[ybar,fill=yellow] coordinates {
%        (Разработанное решение,0.845) (Уровни детализации,0.627)
%      };
%      \addplot[ybar,fill=red] coordinates {
%        (Разработанное решение,0.844) (Уровни детализации,0.740)
%      };
%      \addplot[ybar,fill=blue] coordinates {
%        (Разработанное решение,0.834) (Уровни детализации, 0.640)
%      };
%    \end{axis}
%  \end{tikzpicture}
%  \caption{SSIM-качество отрендереных изображений с трёх ракурсов (обозначены цветами). Больше -- лучше}
%  \label{fig:quality}
%\end{figure}
